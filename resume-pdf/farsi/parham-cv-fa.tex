%%%%%%%%%%%%%%%%%%%%%%%%%%%%%%%%%%%%%%%%%
% In The Name of God
%
%
% Parham Alvani Resume/CV
% parham.alvani@gmail.com
% https://github.com/1995parham
%
%%%%%%%%%%%%%%%%%%%%%%%%%%%%%%%%%%%%%%%%%

%%%%%%%%%%%%%%%%%%%%%%%%%%%%%%%%%%%%%%%%%
% Friggeri Resume/CV
% XeLaTeX Template
% Version 1.2 (3/5/15)
%
% This template has been downloaded from:
% http://www.LaTeXTemplates.com
%
% Original author:
% Adrien Friggeri (adrien@friggeri.net)
% https://github.com/afriggeri/CV
%
% License:
% CC BY-NC-SA 3.0 (http://creativecommons.org/licenses/by-nc-sa/3.0/)
%
% Important notes:
% This template needs to be compiled with XeLaTeX and the bibliography, if used,
% needs to be compiled with biber rather than bibtex.
%
%%%%%%%%%%%%%%%%%%%%%%%%%%%%%%%%%%%%%%%%%

\documentclass[]{friggeri-cv-fa} % Add 'print' as an option into the square bracket to remove colors from this template for printing

\usepackage{parham-cv}
\usepackage{xepersian}

\addbibresource{bibliography.bib} % Specify the bibliography file to include publications

\begin{document}

% Your name and current job title/field
\header{پرهام}{الوانی}{دانشجوی مهندسی نرم افزار}


%----------------------------------------------------------------------------------------
%	SIDEBAR SECTION
%----------------------------------------------------------------------------------------

\begin{aside} % In the aside, each new line forces a line break
\section{\textcolor{TextYellow}{ا}رتباط}
دانشگاه امیرکبیر
خیابان حافظ
تهران، ایران
-
\lr{+98 (939) 090 9540}
\lr{+98 (21) 8802 13311}
-
\begin{latin}
\href{mailto:parham.alvani@gmail.com}{parham.alvani@gmail.com}
\href{mailto:parham.alvani@aut.ac.ir}{parham.alvani@aut.ac.ir}
\href{http://1995parham.github.io}{http://1995parham.github.io}
\end{latin}
\section{\textcolor{TextOrange}{ز}بان‌ها}
فارسی:
زبان مادری
انگلیسی:
تسلط محدود برای کار
\section{\textcolor{TextGreen}{ب}رنامه‌نویسی}
\begin{latin}
{\color{red} $\varheartsuit$} C, Go
Python3, PHP
AMD64 Assembly,
AVR Assembly,
Java SE, Java EE
CSS3 \& HTML5
NodeJs, JS
\end{latin}
\section{\textcolor{DarkBlue}{پ}روژه‌ها}
\begin{latin}
\href{https://github.com/1995parham}{\textcolor{TextGreen}{Github}}
\end{latin}
\section{\textcolor{Ocean}{آخرین}به روز رسانی}
\today
\end{aside}

%----------------------------------------------------------------------------------------
%	INTERESTS SECTION
%----------------------------------------------------------------------------------------

\section{علایق}
\textbf{حرفه‌ای:}
\begin{itemize}
	\item اینترنت اشیا
	\item شبکه‌های نرم افزار بنیان
	\item مجازی سازی توابع شبکه
	\item \lr{Kernel Hacking}
	\item تئوری گراف
	\item آنالیز ریاضی
\end{itemize}
\textbf{شخصی:}
\begin{itemize}	
	\item شطرنج
	\item دویدن
\end{itemize}


%----------------------------------------------------------------------------------------
%	EDUCATION SECTION
%----------------------------------------------------------------------------------------

\section{تحصیلات}

\begin{entrylist}

%------------------------------------------------

\entry
{۱۳۹۲-۱۳۹۶}
{کارشناسی, {\normalfont مهندسی کامپیوتر}}
{دانشگاه صنعتی امیرکبیر}
{معدل کل: ۱۹.۱۸ از ۲۰ (۸۷ واحد)}

%------------------------------------------------

\entry
{۱۳۸۸-۱۳۹۲}
{دبیرستان, {\normalfont دیپلم ریاضی و فیزیک}}
{دبیرستان انرژی اتمی}
{معدل کل: ۱۹.۴۵ از ۲۰}

%------------------------------------------------


\end{entrylist}

%----------------------------------------------------------------------------------------
%	HONORS and AWARDS SECTION
%----------------------------------------------------------------------------------------

\section{جوایز و افتخارات}

\begin{entrylist}

%------------------------------------------------

\entry
{1394}
{{\normalfont فرصت انتخاب} \textcolor{TextGreen}{رشته دوم} {\normalfont به خاطر آمادگی}}
{}
{}

%------------------------------------------------

\entry
{بهار ۱۳۹۴}
{\normalfont Awarded as an Outstanding Student by the head of Computer Engineering and Information Technology department, AmirKabir University of Technology}
{}
{}

%------------------------------------------------

\entry
{بهار ۱۳۹۴}
{{\normalfont Achieved} \textcolor{TextYellow}{3rd} {\normalfont Place in 1st National Digital Design Contest of Iran CoDesign League as a Member of \emph{Polytechnic} team}}
{}
{}

%------------------------------------------------

\entry
{پاییز ۱۳۹۳}
{{\normalfont Achieved} \textcolor{UniBlue}{3rd} {\normalfont Place in 14th Amirkabir ACM International Collegiate Programming Contest (ICPC) as a Member of \emph{703} team}}
{}
{}

%------------------------------------------------

\entry
{پاییز ۱۳۹۳}
{\normalfont Participated in 39th Asia Regional ACM International Collegiate Programming Contest (ICPC) in Sharif University of Technology as a Member of \emph{703} team}
{}
{}

%------------------------------------------------

\entry
{۱۳۹۲}
{\normalfont Ranked in $0.2\%$ among more than $251,956$ participators in Nation-wide University Entrance Exam among all Iranian Students of Math. \& Physics}
{}
{}

%------------------------------------------------

\entry
{۱۳۹۰}
{\textcolor{TextOrange}{Semi-finalist} {\normalfont at National Mathematics and Computer Olympiads}}
{}
{}

%------------------------------------------------

\entry
{۱۳۸۹}
{\textcolor{TextOrange}{Semi-finalist} {\normalfont at National Mathematics and Computer Olympiads}}
{}
{}

%------------------------------------------------

\entry
{۱۳۸۸}
{\textcolor{TextOrange}{Semi-finalist} {\normalfont at National Mathematics and Computer Olympiads}}
{}
{}

%------------------------------------------------

\entry
{۱۳۸۸}
{{\normalfont Achieved} \textcolor{Ocean}{Merit Award} {\normalfont in the International Mathematics Competition (IMC) Icheon, South Korea}}
{}
{}

%------------------------------------------------

\entry
{۱۳۸۵-۱۳۸۸}
{\normalfont Member of National Organization for Development of Exceptional Talents \href{https://en.wikipedia.org/wiki/National_Organization_for_Development_of_Exceptional_Talents}{(NODET)}}
{}
{}


\end{entrylist}

%----------------------------------------------------------------------------------------
%       RESEARCH EXPERIENCE
%----------------------------------------------------------------------------------------

\section{تجربیات تحقیقاتی}

\begin{entrylist}

\entry
{۱۳۹۴-اکنون}
{بستر مدیریت هوشنمند ساختمان بر اساس اینترنت اشیا}
{تحت نظارت دکتر بخشی، دانشکده کامپیوتر دانشگاه امیرکبیر، تهران، ایران}
{تحقیق بر روی بسترها و سیستم عامل‌های اینترنت اشیا}

\end{entrylist}

%----------------------------------------------------------------------------------------
%       SKILLS and EXPERTISE
%----------------------------------------------------------------------------------------

\section{مهارت‌ها}

\begin{entrylist}

\entry
{\textcolor{TextGreen}{$\bullet$}}
{زبان‌های برنامه نویسی تابعی}
{}
{\lr{SML, Lisp, Haskell}}

%------------------------------------------------

\entry
{\textcolor{TextOrange}{$\bullet$}}
{زبان‌های توصیف سخت افزار}
{}
{\lr{Verilog HDL, VHDL, Spice}}

%------------------------------------------------

\entry
{\textcolor{DarkBlue}{$\bullet$}}
{ابزارآلات و پروتکل‌های شبکه‌ای}
{}
{\lr{ONOS Platform, FloodLight, Ryu, Mininet, GNS3, Wireshark, Cisco Packet Tracer, OpenFlow1.3}}

%------------------------------------------------

\entry
{\textcolor{Ocean}{$\bullet$}}
{ابزارآلات تایپ}
{}
{\lr{\LaTeX, Microsoft Word, LibreOffice, Pages, Vim}}

%------------------------------------------------

\entry
{\textcolor{LightGray}{$\bullet$}}
{شبیه‌ساز‌های سخت افزاری}
{}
{\lr{Xilinx Vivado Design Suite, P-Spice, H-Spice, Proteus, Quartus II}}

%------------------------------------------------

\entry
{\textcolor{TextYellow}{$\bullet$}}
{سیستم‌های مدیریت پایگاه داده}
{}
{\lr{MySQL, PostgreSQL, MongoDB}}

%------------------------------------------------

\entry
{\textcolor{TextRed}{$\bullet$}}
{سیستم‌های عامل}
{}
{\lr{Ubuntu, Ubuntu Server, CentOS, OS X, Windows, Windows Server 2012}}

%------------------------------------------------

\entry
{\textcolor{TextPink}{$\bullet$}}
{گوناگون}
{}
{\lr{Netbeans, IntelliJ, Eclipse, Octave, Matlab}}

%------------------------------------------------

\entry
{\textcolor{UniBlue}{$\bullet$}}
{بسترهای برنامه نویسی}
{}
{
	\textbf{\lr{C Frameworks}}
	\begin{itemize}
		\item \lr{CGI Programming}
		\item \lr{BSD Socket Programming}
		\item \lr{UNIX System Programming}
		\item \lr{GTK}
		\item \lr{Gnome}
	\end{itemize}
	
	\textbf{Java Frameworks}
	\begin{itemize}
		\item \lr{Maven}
		\item \lr{Log4j}
		\item \lr{JPA(Java Persistence API)}
		\item \lr{Hibernate}
		\item \lr{JSP}
		\item \lr{JSF}
	\end{itemize}
}

%------------------------------------------------


\end{entrylist}

%----------------------------------------------------------------------------------------
%       TEACHING EXPERIENCES
%----------------------------------------------------------------------------------------

\section{تجربیات تدریس}

\begin{entrylist}

\entry
{پاییز ۱۳۹۳}
{مبانی برنامه نویسی و کامپیوتر}
{تدریسیار}
{دانشگاه صنعتی امیرکبیر تحت نظارت دکتر بخشی}

%------------------------------------------------

\entry
{بهار ۱۳۹۴}
{برنامه نویسی پیشرفته}
{تدریسیار}
{دانشگاه صنعتی امیرکبیر تحت نظارت دکتر نورحسینی}

%------------------------------------------------

\entry
{بهار ۱۳۹۴}
{ریاضیات گسسته}
{تدریسیار}
{دانشگاه صنعتی امیرکبیر تحت نظارت دکتر فلاح}

%------------------------------------------------

\entry
{بهار ۱۳۹۴}
{مقدمه‌ای بر برنامه نویسی پایتون}
{ارائه دهنده}
{کارگاه پایتون، هفتمین جشواره لینوکس}

%------------------------------------------------

\entry
{پاییز ۱۳۹۴}
{مبانی برنامه نویسی و کامپیوتر}
{تدریسیار}
{دانشگاه صنعتی امیرکبیر تحت نظارت دکتر بخشی}

%------------------------------------------------

\entry
{بهار ۱۳۹۵}
{برنامه نویسی پیشرفته}
{تدریسیار}
{دانشگاه صنعتی امیرکبیر تحت نظارت دکتر نورحسینی}

%------------------------------------------------

\entry
{بهار ۱۳۹۵}
{مبانی طراحی پایگاه داده}
{تدریسیار}
{دانشگاه صنعتی امیرکبیر تحت نظارت دکتر ممتازی}

%------------------------------------------------

\entry
{بهار ۱۳۹۵}
{ریزپردازنده ۱}
{تدریسیار}
{دانشگاه صنعتی امیرکبیر تحت نظارت دکتر همایونپور}

%------------------------------------------------

\entry
{بهار ۱۳۹۵}
{آمار و احتمال مهندسی}
{تدریسیار}
{دانشگاه صنعتی امیرکبیر تحت نظارت دکتر امیرحائری}

%------------------------------------------------


\end{entrylist}

%----------------------------------------------------------------------------------------
%       GRADUATE COURSES
%----------------------------------------------------------------------------------------

\section{درس‌های ارشد}

\begin{entrylist}

\entry
{پاییز ۱۳۹۴}
{شبکه‌های کامپیوتری پیشرفته}
{دکتر خرسندی}
{}

%------------------------------------------------

\end{entrylist}

%----------------------------------------------------------------------------------------
%       CERTIFICATIONS
%----------------------------------------------------------------------------------------

\section{مدارک}
\begin{itemize}
	\item \lr{CompTIA Network+}
	\item \lr{Supporting Windows 8.1(70-688)}
	\item \lr{Installing and Configuring Windows Server 2012(70-410)}
	\item \lr{Administering Windows Server 2012(70-411)}
	\item \lr{Configuring Advanced Windows Server 2012 Services(70-412)}
	\item \lr{Designing and Implementing a Server Infrastructure(70-413)}
	\item \lr{Implementing an Advanced Server Infrastructure(70-414)}
	\item \lr{LPIC-1(101-102)}
	\item \lr{LPIC-2(201-202)}
	\item \lr{Cisco CCENT}
\end{itemize}

\end{document}

